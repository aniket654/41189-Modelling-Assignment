\documentclass[11pt]{article}

    \usepackage[breakable]{tcolorbox}
    \usepackage{parskip} % Stop auto-indenting (to mimic markdown behaviour)
    
    \usepackage{iftex}
    \ifPDFTeX
    	\usepackage[T1]{fontenc}
    	\usepackage{mathpazo}
    \else
    	\usepackage{fontspec}
    \fi

    % Basic figure setup, for now with no caption control since it's done
    % automatically by Pandoc (which extracts ![](path) syntax from Markdown).
    \usepackage{graphicx}
    % Maintain compatibility with old templates. Remove in nbconvert 6.0
    \let\Oldincludegraphics\includegraphics
    % Ensure that by default, figures have no caption (until we provide a
    % proper Figure object with a Caption API and a way to capture that
    % in the conversion process - todo).
    \usepackage{caption}
    \DeclareCaptionFormat{nocaption}{}
    \captionsetup{format=nocaption,aboveskip=0pt,belowskip=0pt}

    \usepackage[Export]{adjustbox} % Used to constrain images to a maximum size
    \adjustboxset{max size={0.9\linewidth}{0.9\paperheight}}
    \usepackage{float}
    \floatplacement{figure}{H} % forces figures to be placed at the correct location
    \usepackage{xcolor} % Allow colors to be defined
    \usepackage{enumerate} % Needed for markdown enumerations to work
    \usepackage{geometry} % Used to adjust the document margins
    \usepackage{amsmath} % Equations
    \usepackage{amssymb} % Equations
    \usepackage{textcomp} % defines textquotesingle
    % Hack from http://tex.stackexchange.com/a/47451/13684:
    \AtBeginDocument{%
        \def\PYZsq{\textquotesingle}% Upright quotes in Pygmentized code
    }
    \usepackage{upquote} % Upright quotes for verbatim code
    \usepackage{eurosym} % defines \euro
    \usepackage[mathletters]{ucs} % Extended unicode (utf-8) support
    \usepackage{fancyvrb} % verbatim replacement that allows latex
    \usepackage{grffile} % extends the file name processing of package graphics 
                         % to support a larger range
    \makeatletter % fix for grffile with XeLaTeX
    \def\Gread@@xetex#1{%
      \IfFileExists{"\Gin@base".bb}%
      {\Gread@eps{\Gin@base.bb}}%
      {\Gread@@xetex@aux#1}%
    }
    \makeatother

    % The hyperref package gives us a pdf with properly built
    % internal navigation ('pdf bookmarks' for the table of contents,
    % internal cross-reference links, web links for URLs, etc.)
    \usepackage{hyperref}
    % The default LaTeX title has an obnoxious amount of whitespace. By default,
    % titling removes some of it. It also provides customization options.
    \usepackage{titling}
    \usepackage{longtable} % longtable support required by pandoc >1.10
    \usepackage{booktabs}  % table support for pandoc > 1.12.2
    \usepackage[inline]{enumitem} % IRkernel/repr support (it uses the enumerate* environment)
    \usepackage[normalem]{ulem} % ulem is needed to support strikethroughs (\sout)
                                % normalem makes italics be italics, not underlines
    \usepackage{mathrsfs}
    

    
    % Colors for the hyperref package
    \definecolor{urlcolor}{rgb}{0,.145,.698}
    \definecolor{linkcolor}{rgb}{.71,0.21,0.01}
    \definecolor{citecolor}{rgb}{.12,.54,.11}

    % ANSI colors
    \definecolor{ansi-black}{HTML}{3E424D}
    \definecolor{ansi-black-intense}{HTML}{282C36}
    \definecolor{ansi-red}{HTML}{E75C58}
    \definecolor{ansi-red-intense}{HTML}{B22B31}
    \definecolor{ansi-green}{HTML}{00A250}
    \definecolor{ansi-green-intense}{HTML}{007427}
    \definecolor{ansi-yellow}{HTML}{DDB62B}
    \definecolor{ansi-yellow-intense}{HTML}{B27D12}
    \definecolor{ansi-blue}{HTML}{208FFB}
    \definecolor{ansi-blue-intense}{HTML}{0065CA}
    \definecolor{ansi-magenta}{HTML}{D160C4}
    \definecolor{ansi-magenta-intense}{HTML}{A03196}
    \definecolor{ansi-cyan}{HTML}{60C6C8}
    \definecolor{ansi-cyan-intense}{HTML}{258F8F}
    \definecolor{ansi-white}{HTML}{C5C1B4}
    \definecolor{ansi-white-intense}{HTML}{A1A6B2}
    \definecolor{ansi-default-inverse-fg}{HTML}{FFFFFF}
    \definecolor{ansi-default-inverse-bg}{HTML}{000000}

    % commands and environments needed by pandoc snippets
    % extracted from the output of `pandoc -s`
    \providecommand{\tightlist}{%
      \setlength{\itemsep}{0pt}\setlength{\parskip}{0pt}}
    \DefineVerbatimEnvironment{Highlighting}{Verbatim}{commandchars=\\\{\}}
    % Add ',fontsize=\small' for more characters per line
    \newenvironment{Shaded}{}{}
    \newcommand{\KeywordTok}[1]{\textcolor[rgb]{0.00,0.44,0.13}{\textbf{{#1}}}}
    \newcommand{\DataTypeTok}[1]{\textcolor[rgb]{0.56,0.13,0.00}{{#1}}}
    \newcommand{\DecValTok}[1]{\textcolor[rgb]{0.25,0.63,0.44}{{#1}}}
    \newcommand{\BaseNTok}[1]{\textcolor[rgb]{0.25,0.63,0.44}{{#1}}}
    \newcommand{\FloatTok}[1]{\textcolor[rgb]{0.25,0.63,0.44}{{#1}}}
    \newcommand{\CharTok}[1]{\textcolor[rgb]{0.25,0.44,0.63}{{#1}}}
    \newcommand{\StringTok}[1]{\textcolor[rgb]{0.25,0.44,0.63}{{#1}}}
    \newcommand{\CommentTok}[1]{\textcolor[rgb]{0.38,0.63,0.69}{\textit{{#1}}}}
    \newcommand{\OtherTok}[1]{\textcolor[rgb]{0.00,0.44,0.13}{{#1}}}
    \newcommand{\AlertTok}[1]{\textcolor[rgb]{1.00,0.00,0.00}{\textbf{{#1}}}}
    \newcommand{\FunctionTok}[1]{\textcolor[rgb]{0.02,0.16,0.49}{{#1}}}
    \newcommand{\RegionMarkerTok}[1]{{#1}}
    \newcommand{\ErrorTok}[1]{\textcolor[rgb]{1.00,0.00,0.00}{\textbf{{#1}}}}
    \newcommand{\NormalTok}[1]{{#1}}
    
    % Additional commands for more recent versions of Pandoc
    \newcommand{\ConstantTok}[1]{\textcolor[rgb]{0.53,0.00,0.00}{{#1}}}
    \newcommand{\SpecialCharTok}[1]{\textcolor[rgb]{0.25,0.44,0.63}{{#1}}}
    \newcommand{\VerbatimStringTok}[1]{\textcolor[rgb]{0.25,0.44,0.63}{{#1}}}
    \newcommand{\SpecialStringTok}[1]{\textcolor[rgb]{0.73,0.40,0.53}{{#1}}}
    \newcommand{\ImportTok}[1]{{#1}}
    \newcommand{\DocumentationTok}[1]{\textcolor[rgb]{0.73,0.13,0.13}{\textit{{#1}}}}
    \newcommand{\AnnotationTok}[1]{\textcolor[rgb]{0.38,0.63,0.69}{\textbf{\textit{{#1}}}}}
    \newcommand{\CommentVarTok}[1]{\textcolor[rgb]{0.38,0.63,0.69}{\textbf{\textit{{#1}}}}}
    \newcommand{\VariableTok}[1]{\textcolor[rgb]{0.10,0.09,0.49}{{#1}}}
    \newcommand{\ControlFlowTok}[1]{\textcolor[rgb]{0.00,0.44,0.13}{\textbf{{#1}}}}
    \newcommand{\OperatorTok}[1]{\textcolor[rgb]{0.40,0.40,0.40}{{#1}}}
    \newcommand{\BuiltInTok}[1]{{#1}}
    \newcommand{\ExtensionTok}[1]{{#1}}
    \newcommand{\PreprocessorTok}[1]{\textcolor[rgb]{0.74,0.48,0.00}{{#1}}}
    \newcommand{\AttributeTok}[1]{\textcolor[rgb]{0.49,0.56,0.16}{{#1}}}
    \newcommand{\InformationTok}[1]{\textcolor[rgb]{0.38,0.63,0.69}{\textbf{\textit{{#1}}}}}
    \newcommand{\WarningTok}[1]{\textcolor[rgb]{0.38,0.63,0.69}{\textbf{\textit{{#1}}}}}
    
    
    % Define a nice break command that doesn't care if a line doesn't already
    % exist.
    \def\br{\hspace*{\fill} \\* }
    % Math Jax compatibility definitions
    \def\gt{>}
    \def\lt{<}
    \let\Oldtex\TeX
    \let\Oldlatex\LaTeX
    \renewcommand{\TeX}{\textrm{\Oldtex}}
    \renewcommand{\LaTeX}{\textrm{\Oldlatex}}
    % Document parameters
    % Document title
    \title{41189 Modelling Assignment - Pre-Submission}
    
    
    
    
    
% Pygments definitions
\makeatletter
\def\PY@reset{\let\PY@it=\relax \let\PY@bf=\relax%
    \let\PY@ul=\relax \let\PY@tc=\relax%
    \let\PY@bc=\relax \let\PY@ff=\relax}
\def\PY@tok#1{\csname PY@tok@#1\endcsname}
\def\PY@toks#1+{\ifx\relax#1\empty\else%
    \PY@tok{#1}\expandafter\PY@toks\fi}
\def\PY@do#1{\PY@bc{\PY@tc{\PY@ul{%
    \PY@it{\PY@bf{\PY@ff{#1}}}}}}}
\def\PY#1#2{\PY@reset\PY@toks#1+\relax+\PY@do{#2}}

\expandafter\def\csname PY@tok@gd\endcsname{\def\PY@tc##1{\textcolor[rgb]{0.63,0.00,0.00}{##1}}}
\expandafter\def\csname PY@tok@gu\endcsname{\let\PY@bf=\textbf\def\PY@tc##1{\textcolor[rgb]{0.50,0.00,0.50}{##1}}}
\expandafter\def\csname PY@tok@gt\endcsname{\def\PY@tc##1{\textcolor[rgb]{0.00,0.27,0.87}{##1}}}
\expandafter\def\csname PY@tok@gs\endcsname{\let\PY@bf=\textbf}
\expandafter\def\csname PY@tok@gr\endcsname{\def\PY@tc##1{\textcolor[rgb]{1.00,0.00,0.00}{##1}}}
\expandafter\def\csname PY@tok@cm\endcsname{\let\PY@it=\textit\def\PY@tc##1{\textcolor[rgb]{0.25,0.50,0.50}{##1}}}
\expandafter\def\csname PY@tok@vg\endcsname{\def\PY@tc##1{\textcolor[rgb]{0.10,0.09,0.49}{##1}}}
\expandafter\def\csname PY@tok@vi\endcsname{\def\PY@tc##1{\textcolor[rgb]{0.10,0.09,0.49}{##1}}}
\expandafter\def\csname PY@tok@vm\endcsname{\def\PY@tc##1{\textcolor[rgb]{0.10,0.09,0.49}{##1}}}
\expandafter\def\csname PY@tok@mh\endcsname{\def\PY@tc##1{\textcolor[rgb]{0.40,0.40,0.40}{##1}}}
\expandafter\def\csname PY@tok@cs\endcsname{\let\PY@it=\textit\def\PY@tc##1{\textcolor[rgb]{0.25,0.50,0.50}{##1}}}
\expandafter\def\csname PY@tok@ge\endcsname{\let\PY@it=\textit}
\expandafter\def\csname PY@tok@vc\endcsname{\def\PY@tc##1{\textcolor[rgb]{0.10,0.09,0.49}{##1}}}
\expandafter\def\csname PY@tok@il\endcsname{\def\PY@tc##1{\textcolor[rgb]{0.40,0.40,0.40}{##1}}}
\expandafter\def\csname PY@tok@go\endcsname{\def\PY@tc##1{\textcolor[rgb]{0.53,0.53,0.53}{##1}}}
\expandafter\def\csname PY@tok@cp\endcsname{\def\PY@tc##1{\textcolor[rgb]{0.74,0.48,0.00}{##1}}}
\expandafter\def\csname PY@tok@gi\endcsname{\def\PY@tc##1{\textcolor[rgb]{0.00,0.63,0.00}{##1}}}
\expandafter\def\csname PY@tok@gh\endcsname{\let\PY@bf=\textbf\def\PY@tc##1{\textcolor[rgb]{0.00,0.00,0.50}{##1}}}
\expandafter\def\csname PY@tok@ni\endcsname{\let\PY@bf=\textbf\def\PY@tc##1{\textcolor[rgb]{0.60,0.60,0.60}{##1}}}
\expandafter\def\csname PY@tok@nl\endcsname{\def\PY@tc##1{\textcolor[rgb]{0.63,0.63,0.00}{##1}}}
\expandafter\def\csname PY@tok@nn\endcsname{\let\PY@bf=\textbf\def\PY@tc##1{\textcolor[rgb]{0.00,0.00,1.00}{##1}}}
\expandafter\def\csname PY@tok@no\endcsname{\def\PY@tc##1{\textcolor[rgb]{0.53,0.00,0.00}{##1}}}
\expandafter\def\csname PY@tok@na\endcsname{\def\PY@tc##1{\textcolor[rgb]{0.49,0.56,0.16}{##1}}}
\expandafter\def\csname PY@tok@nb\endcsname{\def\PY@tc##1{\textcolor[rgb]{0.00,0.50,0.00}{##1}}}
\expandafter\def\csname PY@tok@nc\endcsname{\let\PY@bf=\textbf\def\PY@tc##1{\textcolor[rgb]{0.00,0.00,1.00}{##1}}}
\expandafter\def\csname PY@tok@nd\endcsname{\def\PY@tc##1{\textcolor[rgb]{0.67,0.13,1.00}{##1}}}
\expandafter\def\csname PY@tok@ne\endcsname{\let\PY@bf=\textbf\def\PY@tc##1{\textcolor[rgb]{0.82,0.25,0.23}{##1}}}
\expandafter\def\csname PY@tok@nf\endcsname{\def\PY@tc##1{\textcolor[rgb]{0.00,0.00,1.00}{##1}}}
\expandafter\def\csname PY@tok@si\endcsname{\let\PY@bf=\textbf\def\PY@tc##1{\textcolor[rgb]{0.73,0.40,0.53}{##1}}}
\expandafter\def\csname PY@tok@s2\endcsname{\def\PY@tc##1{\textcolor[rgb]{0.73,0.13,0.13}{##1}}}
\expandafter\def\csname PY@tok@nt\endcsname{\let\PY@bf=\textbf\def\PY@tc##1{\textcolor[rgb]{0.00,0.50,0.00}{##1}}}
\expandafter\def\csname PY@tok@nv\endcsname{\def\PY@tc##1{\textcolor[rgb]{0.10,0.09,0.49}{##1}}}
\expandafter\def\csname PY@tok@s1\endcsname{\def\PY@tc##1{\textcolor[rgb]{0.73,0.13,0.13}{##1}}}
\expandafter\def\csname PY@tok@dl\endcsname{\def\PY@tc##1{\textcolor[rgb]{0.73,0.13,0.13}{##1}}}
\expandafter\def\csname PY@tok@ch\endcsname{\let\PY@it=\textit\def\PY@tc##1{\textcolor[rgb]{0.25,0.50,0.50}{##1}}}
\expandafter\def\csname PY@tok@m\endcsname{\def\PY@tc##1{\textcolor[rgb]{0.40,0.40,0.40}{##1}}}
\expandafter\def\csname PY@tok@gp\endcsname{\let\PY@bf=\textbf\def\PY@tc##1{\textcolor[rgb]{0.00,0.00,0.50}{##1}}}
\expandafter\def\csname PY@tok@sh\endcsname{\def\PY@tc##1{\textcolor[rgb]{0.73,0.13,0.13}{##1}}}
\expandafter\def\csname PY@tok@ow\endcsname{\let\PY@bf=\textbf\def\PY@tc##1{\textcolor[rgb]{0.67,0.13,1.00}{##1}}}
\expandafter\def\csname PY@tok@sx\endcsname{\def\PY@tc##1{\textcolor[rgb]{0.00,0.50,0.00}{##1}}}
\expandafter\def\csname PY@tok@bp\endcsname{\def\PY@tc##1{\textcolor[rgb]{0.00,0.50,0.00}{##1}}}
\expandafter\def\csname PY@tok@c1\endcsname{\let\PY@it=\textit\def\PY@tc##1{\textcolor[rgb]{0.25,0.50,0.50}{##1}}}
\expandafter\def\csname PY@tok@fm\endcsname{\def\PY@tc##1{\textcolor[rgb]{0.00,0.00,1.00}{##1}}}
\expandafter\def\csname PY@tok@o\endcsname{\def\PY@tc##1{\textcolor[rgb]{0.40,0.40,0.40}{##1}}}
\expandafter\def\csname PY@tok@kc\endcsname{\let\PY@bf=\textbf\def\PY@tc##1{\textcolor[rgb]{0.00,0.50,0.00}{##1}}}
\expandafter\def\csname PY@tok@c\endcsname{\let\PY@it=\textit\def\PY@tc##1{\textcolor[rgb]{0.25,0.50,0.50}{##1}}}
\expandafter\def\csname PY@tok@mf\endcsname{\def\PY@tc##1{\textcolor[rgb]{0.40,0.40,0.40}{##1}}}
\expandafter\def\csname PY@tok@err\endcsname{\def\PY@bc##1{\setlength{\fboxsep}{0pt}\fcolorbox[rgb]{1.00,0.00,0.00}{1,1,1}{\strut ##1}}}
\expandafter\def\csname PY@tok@mb\endcsname{\def\PY@tc##1{\textcolor[rgb]{0.40,0.40,0.40}{##1}}}
\expandafter\def\csname PY@tok@ss\endcsname{\def\PY@tc##1{\textcolor[rgb]{0.10,0.09,0.49}{##1}}}
\expandafter\def\csname PY@tok@sr\endcsname{\def\PY@tc##1{\textcolor[rgb]{0.73,0.40,0.53}{##1}}}
\expandafter\def\csname PY@tok@mo\endcsname{\def\PY@tc##1{\textcolor[rgb]{0.40,0.40,0.40}{##1}}}
\expandafter\def\csname PY@tok@kd\endcsname{\let\PY@bf=\textbf\def\PY@tc##1{\textcolor[rgb]{0.00,0.50,0.00}{##1}}}
\expandafter\def\csname PY@tok@mi\endcsname{\def\PY@tc##1{\textcolor[rgb]{0.40,0.40,0.40}{##1}}}
\expandafter\def\csname PY@tok@kn\endcsname{\let\PY@bf=\textbf\def\PY@tc##1{\textcolor[rgb]{0.00,0.50,0.00}{##1}}}
\expandafter\def\csname PY@tok@cpf\endcsname{\let\PY@it=\textit\def\PY@tc##1{\textcolor[rgb]{0.25,0.50,0.50}{##1}}}
\expandafter\def\csname PY@tok@kr\endcsname{\let\PY@bf=\textbf\def\PY@tc##1{\textcolor[rgb]{0.00,0.50,0.00}{##1}}}
\expandafter\def\csname PY@tok@s\endcsname{\def\PY@tc##1{\textcolor[rgb]{0.73,0.13,0.13}{##1}}}
\expandafter\def\csname PY@tok@kp\endcsname{\def\PY@tc##1{\textcolor[rgb]{0.00,0.50,0.00}{##1}}}
\expandafter\def\csname PY@tok@w\endcsname{\def\PY@tc##1{\textcolor[rgb]{0.73,0.73,0.73}{##1}}}
\expandafter\def\csname PY@tok@kt\endcsname{\def\PY@tc##1{\textcolor[rgb]{0.69,0.00,0.25}{##1}}}
\expandafter\def\csname PY@tok@sc\endcsname{\def\PY@tc##1{\textcolor[rgb]{0.73,0.13,0.13}{##1}}}
\expandafter\def\csname PY@tok@sb\endcsname{\def\PY@tc##1{\textcolor[rgb]{0.73,0.13,0.13}{##1}}}
\expandafter\def\csname PY@tok@sa\endcsname{\def\PY@tc##1{\textcolor[rgb]{0.73,0.13,0.13}{##1}}}
\expandafter\def\csname PY@tok@k\endcsname{\let\PY@bf=\textbf\def\PY@tc##1{\textcolor[rgb]{0.00,0.50,0.00}{##1}}}
\expandafter\def\csname PY@tok@se\endcsname{\let\PY@bf=\textbf\def\PY@tc##1{\textcolor[rgb]{0.73,0.40,0.13}{##1}}}
\expandafter\def\csname PY@tok@sd\endcsname{\let\PY@it=\textit\def\PY@tc##1{\textcolor[rgb]{0.73,0.13,0.13}{##1}}}

\def\PYZbs{\char`\\}
\def\PYZus{\char`\_}
\def\PYZob{\char`\{}
\def\PYZcb{\char`\}}
\def\PYZca{\char`\^}
\def\PYZam{\char`\&}
\def\PYZlt{\char`\<}
\def\PYZgt{\char`\>}
\def\PYZsh{\char`\#}
\def\PYZpc{\char`\%}
\def\PYZdl{\char`\$}
\def\PYZhy{\char`\-}
\def\PYZsq{\char`\'}
\def\PYZdq{\char`\"}
\def\PYZti{\char`\~}
% for compatibility with earlier versions
\def\PYZat{@}
\def\PYZlb{[}
\def\PYZrb{]}
\makeatother


    % For linebreaks inside Verbatim environment from package fancyvrb. 
    \makeatletter
        \newbox\Wrappedcontinuationbox 
        \newbox\Wrappedvisiblespacebox 
        \newcommand*\Wrappedvisiblespace {\textcolor{red}{\textvisiblespace}} 
        \newcommand*\Wrappedcontinuationsymbol {\textcolor{red}{\llap{\tiny$\m@th\hookrightarrow$}}} 
        \newcommand*\Wrappedcontinuationindent {3ex } 
        \newcommand*\Wrappedafterbreak {\kern\Wrappedcontinuationindent\copy\Wrappedcontinuationbox} 
        % Take advantage of the already applied Pygments mark-up to insert 
        % potential linebreaks for TeX processing. 
        %        {, <, #, %, $, ' and ": go to next line. 
        %        _, }, ^, &, >, - and ~: stay at end of broken line. 
        % Use of \textquotesingle for straight quote. 
        \newcommand*\Wrappedbreaksatspecials {% 
            \def\PYGZus{\discretionary{\char`\_}{\Wrappedafterbreak}{\char`\_}}% 
            \def\PYGZob{\discretionary{}{\Wrappedafterbreak\char`\{}{\char`\{}}% 
            \def\PYGZcb{\discretionary{\char`\}}{\Wrappedafterbreak}{\char`\}}}% 
            \def\PYGZca{\discretionary{\char`\^}{\Wrappedafterbreak}{\char`\^}}% 
            \def\PYGZam{\discretionary{\char`\&}{\Wrappedafterbreak}{\char`\&}}% 
            \def\PYGZlt{\discretionary{}{\Wrappedafterbreak\char`\<}{\char`\<}}% 
            \def\PYGZgt{\discretionary{\char`\>}{\Wrappedafterbreak}{\char`\>}}% 
            \def\PYGZsh{\discretionary{}{\Wrappedafterbreak\char`\#}{\char`\#}}% 
            \def\PYGZpc{\discretionary{}{\Wrappedafterbreak\char`\%}{\char`\%}}% 
            \def\PYGZdl{\discretionary{}{\Wrappedafterbreak\char`\$}{\char`\$}}% 
            \def\PYGZhy{\discretionary{\char`\-}{\Wrappedafterbreak}{\char`\-}}% 
            \def\PYGZsq{\discretionary{}{\Wrappedafterbreak\textquotesingle}{\textquotesingle}}% 
            \def\PYGZdq{\discretionary{}{\Wrappedafterbreak\char`\"}{\char`\"}}% 
            \def\PYGZti{\discretionary{\char`\~}{\Wrappedafterbreak}{\char`\~}}% 
        } 
        % Some characters . , ; ? ! / are not pygmentized. 
        % This macro makes them "active" and they will insert potential linebreaks 
        \newcommand*\Wrappedbreaksatpunct {% 
            \lccode`\~`\.\lowercase{\def~}{\discretionary{\hbox{\char`\.}}{\Wrappedafterbreak}{\hbox{\char`\.}}}% 
            \lccode`\~`\,\lowercase{\def~}{\discretionary{\hbox{\char`\,}}{\Wrappedafterbreak}{\hbox{\char`\,}}}% 
            \lccode`\~`\;\lowercase{\def~}{\discretionary{\hbox{\char`\;}}{\Wrappedafterbreak}{\hbox{\char`\;}}}% 
            \lccode`\~`\:\lowercase{\def~}{\discretionary{\hbox{\char`\:}}{\Wrappedafterbreak}{\hbox{\char`\:}}}% 
            \lccode`\~`\?\lowercase{\def~}{\discretionary{\hbox{\char`\?}}{\Wrappedafterbreak}{\hbox{\char`\?}}}% 
            \lccode`\~`\!\lowercase{\def~}{\discretionary{\hbox{\char`\!}}{\Wrappedafterbreak}{\hbox{\char`\!}}}% 
            \lccode`\~`\/\lowercase{\def~}{\discretionary{\hbox{\char`\/}}{\Wrappedafterbreak}{\hbox{\char`\/}}}% 
            \catcode`\.\active
            \catcode`\,\active 
            \catcode`\;\active
            \catcode`\:\active
            \catcode`\?\active
            \catcode`\!\active
            \catcode`\/\active 
            \lccode`\~`\~ 	
        }
    \makeatother

    \let\OriginalVerbatim=\Verbatim
    \makeatletter
    \renewcommand{\Verbatim}[1][1]{%
        %\parskip\z@skip
        \sbox\Wrappedcontinuationbox {\Wrappedcontinuationsymbol}%
        \sbox\Wrappedvisiblespacebox {\FV@SetupFont\Wrappedvisiblespace}%
        \def\FancyVerbFormatLine ##1{\hsize\linewidth
            \vtop{\raggedright\hyphenpenalty\z@\exhyphenpenalty\z@
                \doublehyphendemerits\z@\finalhyphendemerits\z@
                \strut ##1\strut}%
        }%
        % If the linebreak is at a space, the latter will be displayed as visible
        % space at end of first line, and a continuation symbol starts next line.
        % Stretch/shrink are however usually zero for typewriter font.
        \def\FV@Space {%
            \nobreak\hskip\z@ plus\fontdimen3\font minus\fontdimen4\font
            \discretionary{\copy\Wrappedvisiblespacebox}{\Wrappedafterbreak}
            {\kern\fontdimen2\font}%
        }%
        
        % Allow breaks at special characters using \PYG... macros.
        \Wrappedbreaksatspecials
        % Breaks at punctuation characters . , ; ? ! and / need catcode=\active 	
        \OriginalVerbatim[#1,codes*=\Wrappedbreaksatpunct]%
    }
    \makeatother

    % Exact colors from NB
    \definecolor{incolor}{HTML}{303F9F}
    \definecolor{outcolor}{HTML}{D84315}
    \definecolor{cellborder}{HTML}{CFCFCF}
    \definecolor{cellbackground}{HTML}{F7F7F7}
    
    % prompt
    \makeatletter
    \newcommand{\boxspacing}{\kern\kvtcb@left@rule\kern\kvtcb@boxsep}
    \makeatother
    \newcommand{\prompt}[4]{
        \ttfamily\llap{{\color{#2}[#3]:\hspace{3pt}#4}}\vspace{-\baselineskip}
    }
    

    
    % Prevent overflowing lines due to hard-to-break entities
    \sloppy 
    % Setup hyperref package
    \hypersetup{
      breaklinks=true,  % so long urls are correctly broken across lines
      colorlinks=true,
      urlcolor=urlcolor,
      linkcolor=linkcolor,
      citecolor=citecolor,
      }
    % Slightly bigger margins than the latex defaults
    
    \geometry{verbose,tmargin=1in,bmargin=1in,lmargin=1in,rmargin=1in}
    
    

\begin{document}
    
    \maketitle
    
    

    
    \textbf{41189 Modelling Assignment (Pre-Submission) --- Group 5}

Topic: Strategies to contain COVID-19

Group members:

\begin{itemize}
\tightlist
\item
  Akash Sreekumar
\item
  Aniket Dhiman
\item
  Ashraya Nepal
\item
  Redah Sheriff
\end{itemize}

    \hypertarget{introduction}{%
\section{1.0 Introduction}\label{introduction}}

    \hypertarget{state-the-problem-why-it-is-importantrelevant-to-you-andor-more-broadly-to-society}{%
\subsection{1.1 State the problem, why it is important/relevant to you
and/or more broadly to
society}\label{state-the-problem-why-it-is-importantrelevant-to-you-andor-more-broadly-to-society}}

    COVID-19 is a fast growing respiratory virus which has affected over
26,000 people and resulted in 859 deaths in Australia ( Department of
Health, 2020). This virus is unique as it is able to spread from person
to person without an individual knowing that they have it as there is an
incubation period of up to two weeks where individuals may not present
any symptoms, and if they do present symptoms they are usually
disregarded due to it being similar to the common cold (Tasmanian
Government, 2020).

    COVID-19 is not only an issue in Australia but is also listed as a
global pandemic which has caused changes in many livelihoods. With no
vaccine being currently available and the virus's fast spreading, our
group found it vital to understand the problems presented with
containing COVID-19. Although health concerns are a main priority when
it comes to managing the virus, it has also resulted in lockdown laws,
social distancing laws, increase in unemployment rates and has caused a
negative impact on the economy. The virus forces governments to
prioritise either health and safety or economic stability. By testing
particular strategies through chosen models, our group hopes to select
the most effective strategies in order to present a method to reduce the
rate of transmission whilst considering the importance of economic
stability within Australian society and the rest of the world.

    \hypertarget{context-background-information-justification}{%
\section{2.0 Context, background information,
justification}\label{context-background-information-justification}}

    \hypertarget{research-and-review-literature-news-digital-media-drivers-and-how-the-problem-is-being-addressed-by-academia-industry-government-etc.}{%
\subsection{2.1 Research and review literature, news, digital media,
drivers and how the problem is being addressed by academia, industry,
government,
etc.}\label{research-and-review-literature-news-digital-media-drivers-and-how-the-problem-is-being-addressed-by-academia-industry-government-etc.}}

    Gradually evolving into a global pandemic, the transmission of COVID-19
has infested small and large chains throughout the world, urging action
to be implemented for its containment. Whilst China has demonstrated its
rapid and successful strategies of quarantine, social distancing and
isolation, these methods have encouraged countries to adhere to similar
protocols. After careful research, personal, rather than government
action has proven to be the most effective strategy to undergo as
individual behaviours such as washing hands regularly and seeking
medical action are crucial in containing the virus.

    The World Health Organisation (WHO) stated that ``every person has a
responsibility to know the level of transmission locally, and understand
what they can do to protect themselves'', ultimately emphasising the
significance of individual action. With the WHO Outbreak Communications
Planning Guide suggesting that behavioural changes can reduce the spread
by as much as 80\%, governments can tailor society's behaviour in the
right direction and therefore utilise resources for containing the
disease far more effectively (World Health Organisation, 2020). Changes
to human behaviour have not only been effective today, but have played
an important role in the history of outbreaks and diseases, where it was
necessary to slow down and control epidemics such as the Spanish Flu of
1919 and the Ebola Outbreak in 2001 (The Conversation, 2020).

    As people's behaviour can be influenced by social norms, containing the
spread of coronavirus is dependent on effective public health messages
which reinforce positive norms and prohibit the chances of `fake news'
(Nature human behaviour,2020) that is often proliferated widely on
social media and the news. Recently, several messages have been conveyed
through evidence based ways to encourage individuals to engage in
protective behaviours such as wearing face masks and limiting travel, as
credible sources are more persuasive and reassuring (Nature human
behaviour, 2020). Additionally, the Victorian Government has delivered
effective television advertisements which stress the importance of
benefits to the recipient and others, by using language such as ``wash
your hands to protect your parents and grandparents'' (Nature human
behaviour, 2020), most of which have aligned with individual's moral
values. Other communication models have encouraged protective behaviours
through twitter videos of ``covid survivors'' explaining their stories
of hardship (Victoria State Government, 2020).

    Aside from instigating changes to human behaviour, government policies
and strategies have been vital in containing the virus and controlling
any clusters or explosive outbreaks. The Australian government has aimed
to create a situation where society is `living with the disease with the
least possible number of restrictions' (Australian Government, 2020)
where state and territory governments understands the emotional and
physical toll that has been brought about by previous lockdowns and
restrictions. Enhanced flu vaccine programs have been brought forward,
along with maximum capacity restrictions in social and compulsory.
Additionally, there has been a focus on maintaining economic growth by
supporting businesses, education and the workforce. Businesses have
received Employment Wage subsidy schemes, Pandemic Unemployment Payments
and extended loans to assist in maintaining their businesses for the
next few months (Australian Government, 2020).

    \hypertarget{identify-the-system-stakeholders.-for-this-you-will-need-to-construct-a-rich-picture-that-captures-the-essential-elements.}{%
\subsection{2.2 Identify the system + stakeholders. For this, you will
need to construct a ``rich picture'' that captures the essential
elements.}\label{identify-the-system-stakeholders.-for-this-you-will-need-to-construct-a-rich-picture-that-captures-the-essential-elements.}}

    This `rich picture' encompasses the key stakeholders that are relevant
to containing the spread of COVID 19. There are four main key
stakeholders, and two sub stakeholders which each play a significant
role in how we tackle this challenging issue as a nation. The role
individuals play is outlined by physical changes we make to our
behaviour, such as wearing masks and limiting the amount we travel. This
is similar to how businesses' have adapted to containing the spread of
COVID, as workplaces have encouraged staff to work from home to avoid
unnecessary travel, along with rearranging office layouts to align with
social distancing protocols. The government is a key stakeholder, and
can be divided into NSW health and NSW police force. NSW health carries
out health regulations regarding travel, capacity restrictions, and
local testing, while NSW police force ensures all protocols are followed
and any interstate travel is restricted. The Research and Development
teams are the final and most crucial to containing covid-19, as their
findings determine what regulations need to /be followed, how businesses
will run and what precautions individuals need to take. They are a key
stakeholder which bind all elements in an attempt to contain the virus.

    \begin{figure}
\centering
\includegraphics{attachment:image4.png}
\caption{image4.png}
\end{figure}

    \hypertarget{identify-current-models-or-prevailing-frames-that-are-being-used-to-address-the-problem}{%
\subsection{2.3 Identify current models or prevailing frames that are
being used to address the
problem}\label{identify-current-models-or-prevailing-frames-that-are-being-used-to-address-the-problem}}

    A current strategy used to contain Covid-19 worldwide is social
distancing and quarantine. These strategies have been implemented in
order to reduce Covid-19's airborne mode of transmission. Some models
used to frame the data for containing Covid-19 includes the SEIHRD
Dynamical model and the IHME model.

    The SEIHRD (Susceptible-Exposed-Infected-Hospitalised-Recovered-Dead)
dynamical model is a mathematical model which represents the dynamic of
different agents involved in a population affected by the disease
(Felicioni \& Pazos, 2020). This type of model aims to predict crucial
future values such as the maximum number of people that will be affected
by the disease and when that will occur. This model is based on the SEIR
(Susceptible-Exposed-Infected-Dead) model which was used earlier but was
scrapped due to the model assuming that exposed people have been
infected but are not able to transmit the virus before the incubation
period, which outputs flawed results. In addition, the SEIR model
disregarded the critical issue of the number of infected people who need
hospitalisation, because public policies need to keep this number below
the capacity of the health care system in order to prevent its collapse.

    The IHME (Institution of Health Metrics and Evaluation) model
incorporates a hybrid modelling approach, utilising elements of
statistical and disease transmission models. This model uses real-time
data instead of assumptions of how the disease will spread (HealthData,
2020). This results in frequent updates in accordance with data changes
as they occur. The purpose of the IHME model is to determe the extent
and timing of deaths and excess demand for hospital services firstly
within the US, then to later extend to different countries. ``Therefore,
the model does not capture the epidemic's transmission dynamics, and
focuses only on forecasting the death rate and the hospitalization
demand inferred from it`` (Eker, 2020). Thus, the model allows for
policymaking with the data that it is able to forecast, allowing the
government to strategies methods of containing the Codvid-19 virus.

    \hypertarget{discuss-how-prevailing-modelsframes-are-preventing-the-problem-from-being-fully-addressed-or-solved}{%
\subsection{2.4 Discuss how prevailing models/frames are preventing the
problem from being fully addressed or
solved}\label{discuss-how-prevailing-modelsframes-are-preventing-the-problem-from-being-fully-addressed-or-solved}}

    A limitation of the SEIHRD dynamic model assumes homogenous mixing,
meaning that all hosts have identical rates of disease-causing contacts
with a homogenous nature. ``The assumption of a homogeneously mixing
population is often sufficient to obtain general insights once an
epidemic is well established in a population. However, there can be
significant differences in the early stages of an epidemic and in the
final epidemic size. In particular, homogeneous mixing can lead to an
overestimation of the final epidemic size and the magnitude of the
interventions needed to stop an epidemic (Del Valle et al., 2014). This
prevents the SEIHRD dynamic model from fully addressing the problem of
containing the COVID-19 virus as there may be inconsistencies in the
outcome if the SEIHRD dynamic model assumes homogenous mixing.

    Renowned epidemiologist Ruth Etzioni states that the IHME model keeps
``the IHME model keeps changing is evidence of its lack of reliability
as a predictive tool'' (Begley, 2020) and that ``it it is being used for
policy decisions and its results interpreted wrongly is a travesty
unfolding before our eyes.'' Experts say that the IHME model should not
be used as a basis for policymaking due to the fact that the model
projects that deaths will rapidly decline after the peak, thus this
underestimates the severity of the pandemic and may allow the general
public to become complacent in terms of the strategies to contain the
Covid-19 virus such as social distancing and quarantine.

    \hypertarget{conceptual-models}{%
\section{3.0 Conceptual Models}\label{conceptual-models}}

    \hypertarget{based-on-your-self-study-of-the-supermarket-of-models-explain-the-three-models-that-you-have-chosen-to-look-at-the-problem}{%
\subsection{3.1 Based on your self-study of the supermarket of models,
explain the three models that you have chosen to look at the
problem}\label{based-on-your-self-study-of-the-supermarket-of-models-explain-the-three-models-that-you-have-chosen-to-look-at-the-problem}}

    The first model that will be used to understand strategies to contain
COVID-19 is agent-based modelling. Agent based modelling is a
computerised modern simulation that is able to capture the behaviours of
agents within a set environment and has been utilised in the fields of
business through to biology to help solve complex problems. There are
three main components of agent-based modelling which include agents,
environment and time. Agents are known as autonomous elements that can
be in the form of individuals, animals, societies etc (UTS,2020). This
form of modelling monitors chosen agents' interactions with an
environment and its inputted parameters. Hence this creates accurate
simulations allowing for predictions to be made and access emerging
phenomena on a system through multiple points of views and interactions.
An already existing agent-based model that already addresses COVID-19 is
the `COVID-19 virus spread model' that was created by Paul Smaldino.
This agent-based model analyses trends such as the transmission rate
through various parameters, therefore allowing for researchers to
predict and implement recommendations to the public (Martin, 2020).

    The second model that will be useful for our group's investigation is
the behavioural model. Founded by Albert Bandura, the main purpose of
this type of modelling is to observe an individual's or collective
groups behaviour within an environment. It can model both simple and
complex behaviours but requires a large sample size to be analysed.
However, this model benefits from being able to identify certain biases
and heuristics within a given sample. In behavioural modelling it is
important not to assume that the subjects are rational, much rather that
they are irrational. Through repetitive observations, researchers can
form a model that identifies certain characteristics and patterns. By
examining these trends, particular behaviours in given settings can be
seen and various predictions can be inferred before an action occurs
(Husband\& Chong,2011).

    The third model which will be useful for the group's investigation is
the Colonel Blotto Game. The game was first introduced through war where
two colonels were faced against each other. The winner of the battle was
determined by whichever colonel was able to win on the most battle
fronts. This was determined by the number of resources sent to battle at
the fronts by each colonel. Being a strategic game, it is also possible
for the colonel with a lower amount of resources to win against another
with very many resources. An undermanned army can beat an army with
plenty of resources by adding more fronts to the battlefield, more
dimensions, so the resourceful army is more likely to weaken their
troops at each front. Although this was mainly applied during war, it
has made a re-emergence in society as observed through political, social
and many more situations. Some examples of the Colonel Blotto game in
society include politics, sports, hiring and people against their time.

    \hypertarget{explain-how-each-of-the-three-models-works-brief-and-how-it-will-be-applied-in-your-case.}{%
\subsection{3.2 Explain how each of the three models works (brief) and
how it will be applied in your
case.}\label{explain-how-each-of-the-three-models-works-brief-and-how-it-will-be-applied-in-your-case.}}

    Agent-based modelling works through a computerised process where
artificial agents interact within a given model space. Just like
professor Joshua Epstein said, ``if you grow the phenomenon, you'll
understand how it works'', by inputting certain parameters researchers
can create simulations, analyse specific interactions and behaviours of
agents. By investigating these results, researchers can obtain
predictions of certain events and create solutions in real life. In our
chosen topic of `strategies to contain COVID-19', our group decided to
apply and adapt an already existing agent-based model called `COVID-19
Virus spread model' developed by Paul Smaldino. This model currently
allows for parameters such as population size, number of infected
patients, quarantine efforts, healthcare capacity etc to be controlled
(Martin,2020). In our case we would adapt this model to implement and
test our own strategies chosen to tackle the virus. This would include
testing environmental parameters such as isolation, social distancing
and various other hygiene practices. In addition to this we would also
manipulate the agents, in this case to the Australian population. With
these model inputs our group should be able to attain graphs and
statistics about transmission and mortality rates. Hence this will allow
for us to see the most effective strategies to contain COVID-19 and
allow for recommendations to be made for the public.

    As suggested by psychologist Albert Bandura, behavioural modelling
allows for researchers to observe and understand new behaviours
(Husband\& Chong,2011). . This model works through the observation of a
large sample size of agents and creating trends of behaviours. This
allows for predictions and a detailed understanding of what influences
certain behaviours in various settings. In our case study of `Strategies
to contain COVID-19' this model can be utilised by implementing our
group's strategies and observing how individuals react to these
strategies and predominantly its effect on the transmission rate of the
virus. By understanding the particular behaviours and tendencies of the
virus while implementing the strategies our group will be able conclude
which strategies have effect within society, hence efficiently
preventing the spread of COVID-19.

    The Colonel Blotto Game was first used during war. The situation of war
required opposing colonels to strategise against each other to win as
many of the battle fronts as possible using their limited resources. The
colonel that had won the most fronts would win the war. This model can
also be applied to our groups' topic of `strategies to contain
COVID-19'. However, instead of two colonels going against each other it
is the government trying to contain the coronavirus. Through the use of
the game/model, the government has to be able to effectively use its
resources in a manner which decreases the number of coronavirus cases in
their country. In the case of the Australian government, a lot of
resources have been used in an attempt to eradicate the coronavirus such
as establishing social distancing laws, Job keeper/Job seeker payments
in response to the high unemployment rates and continuing search for a
vaccine. All of this is being done while trying to prevent a recession
in the economy. Evidently, Australia has not been perfect in doing these
things. Using the Colonel Blotto game can assist in determining where
most of the government's resources should be going in order to win the
most `fronts' against COVID-19.

    \hypertarget{describe-any-data-sources-real-or-hypothetical-that-are-needed-to-construct-the-three-models.-use-illustrative-drawings-and-diagrams-where-possible}{%
\subsection{3.3 Describe any data sources (real or hypothetical) that
are needed to construct the three models. Use illustrative drawings and
diagrams where
possible}\label{describe-any-data-sources-real-or-hypothetical-that-are-needed-to-construct-the-three-models.-use-illustrative-drawings-and-diagrams-where-possible}}

    For our models to be successfully constructed there are a range of data
sources that are required. Firstly, a source that our models require is
the current strategies that are being implemented and its effect. By
understanding the range of strategies that currently exist across the
world, our group can test the practicality of each strategy (shown in
source 1.0) through our models and provide recommendations of which are
actually most effective. In addition to this our group will be able to
create modification to already existing strategies and test our own
strategies. This can be utilised greatly for both the behavioural model
and Agent- based models (West,Michie, Amlot \& Rubin, 2020)
\includegraphics{attachment:image3.png} Source 1.0 (West,Michie, Amlot
\& Rubin, 2020)

    It is important to obtain general data on the number of coronavirus
cases in each country. This will help in comparing the strategies of
different countries and how effective they are relative to how many
cases they have. Data shows that there have been over 32 million
coronavirus cases worldwide, with USA, India and Brazil being the top
three countries with the most cases (Worldometer, 2020). This can also
be further backed through data on number of coronavirus deaths versus
number of coronavirus recoveries. As shown in the same source, globally
there have been over 900,000 deaths and over 23 million recovered cases.
This gives a ratio of almost 1/23 cases resulting in deaths
(Worldometer, 2020). Through the application of the Colonel Blotto game,
we can determine the `fronts' (countries/states/cities) in which the
coronavirus is emerging and then sending available resources there to
lower the risk.

    Source 1.1 is another integral data source for our modelling as it
highlights vital information of the current statistics of COVID-19 in
Australia. Data which contain statistics of Australia's current state
can be vital specifically for the Agent-based model as real time data
can be used and trialled. Hence this allows for more accurate results to
be obtained. In addition to this, data types such as ``Overseas acquired
last 7 days'' and ``locally acquired'' ( as shown in source 1.2) allow
for specific behavioural patterns to be targeted and analysed more
closely through the behavioural model (Australian Government Department
of Health, 2020).

\includegraphics{attachment:image1.png} Source 1.1 (Australian
Government Department of Health, 2020)
\includegraphics{attachment:image2.png} Source 1.2 (Australian
Government Department of Health, 2020)

    \hypertarget{references}{%
\subsection{4.0 References}\label{references}}

    Australian Government Department of Health. (2020). Coronavirus
(COVID-19) current situation and case numbers. Retrieved 18th September
2020, from
https://www.health.gov.au/news/health-alerts/novel-coronavirus-2019-ncov-health-alert/coronavirus-covid-19-current-situation-and-case-numbers

Australian Government. (2020). Government response to the COVID-19
outbreak.
https://www.health.gov.au/news/health-alerts/novel-coronavirus-2019-ncov-health-alert/government-response-to-the-covid-19-outbreak

Begley, S. (2020). Influential Covid-19 model shouldn't guide U.S.
policies, critics say. Retrieved 21 September 2020, from
https://www.statnews.com/2020/04/17/influential-covid-19-model-uses-flawed-methods-shouldnt-guide-policies-critics-say/

Castilla-Rho, J. (2020). Week 8 - Strategy and Competition: Colonel
Blotto Game (P2) {[}Subject 41189 lecture notes{]}. UTS Canvas.
https://canvas.uts.edu.au/

COVID-19 model FAQs. Institute for Health Metrics and Evaluation.
(2020). Retrieved 25 September 2020, from
http://www.healthdata.org/covid/faqs\#differences\%20in\%20modeling

Del Valle, S., Hyman, J., \& Chitnis, N. (2014). MATHEMATICAL MODELS OF
CONTACT PATTERNS BETWEEN AGE GROUPS FOR PREDICTING THE SPREAD OF
INFECTIOUS DISEASES. Retrieved 21 September 2020, from
https://www.ncbi.nlm.nih.gov/pmc/articles/PMC4002176/

Eker, S. (2020). Validity and usefulness of COVID-19 models. Retrieved
21 September 2020, from
https://www.nature.com/articles/s41599-020-00553-4

Husband, J., Chong, I. (2011). Behaviour Modelling. In Encyclopedia of
Child Behaviour and Development. Retrieved September 25, 2020, from
https://link.springer.com/referenceworkentry/10.1007\%2F978-0-387-79061-9\_307

Martin, N. (2020). COVID-19 VIRUS SPREAD, by Nich Martin (model ID 6282)
-- NetLogo Modeling Commons. Modelingcommons.org. Retrieved 15 September
2020, from
http://modelingcommons.org/browse/one\_model/6282\#model\_tabs\_browse\_info.

Nature Human Behaviour. (2020). Using social and behavioural sciences to
support COVID-19 pandemic response.
https://www.nature.com/articles/s41562-020-0884-z

Nature Medicine. (2020). Modelling the COVID-19 epidemic and
implementation of population --wide interventions in Italy.
https://www.nature.com/articles/s41591-020-0883-7

Pazos, F., \& Felicioni, F. (2020). A control approach to the Covid-19
disease using a SEIHRD dynamical model. Medrxiv.org. Retrieved 21
September 2020, from
https://www.medrxiv.org/content/10.1101/2020.05.27.20115295v1.full.pdf

Tasmanian Government. (2020). About coronaviruses and COVID-19.
Retrieved 23 September 2020, from
https://coronavirus.tas.gov.au/facts/about-coronaviruses-and-covid19.

The Government of Ireland. (2019). Briefing on the Australian
government's response to COVID-19 -- Monday 21 September 2020.
https://www.gov.ie/en/publication/388d7-briefing-on-the-governments-response-to-covid-19-monday-21st-september-2020/

The LANCET. (2020). How will country-based mitigation measures influence
the course of the COVID-19 epidemic?.
https://www.thelancet.com/journals/lancet/article/PIIS0140-6736(20)30567-5/fulltext\#fig1

University of Technology Sydney. (2020). Agent-based modelling.
Retrieved 18 September 2020, from
https://www.uts.edu.au/about/faculty-engineering-and-information-technology/information-systems-and-modelling/research/centre-persuasive-systems-wise-adaptive-living-perswade/perswade-streams/agent-based-modelling.

West.R, M Susan, Rubin J \& Amlot R. (2020). Applying principles of
behaviour change to reduce SARS-CoV-2 transmission. Nature Human
behaviour. Retrieved 22nd September 2020 from World Health Organisation.
(2020). COVID-19 Strategy Update.
https://www.who.int/docs/default-source/coronaviruse/covid-strategy-update-14april2020.pdf

World Health Organisation. (2020). Director-General's opening remarks at
the media on COVID-19.
https://www.who.int/dg/speeches/detail/who-director-general-s-opening-remarks-at-the-media-briefing-on-covid-19---21-august-2020

Worldometer. (2020, September 25). Coronavirus.
https://www.worldometers.info/coronavirus/?utm\_campaign=homeAdUOA?Si


    % Add a bibliography block to the postdoc
    
    
    
\end{document}
